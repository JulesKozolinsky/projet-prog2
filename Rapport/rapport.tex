\documentclass{article}
\usepackage[utf8]{inputenc}
\usepackage[francais]{babel}
\usepackage[T1]{fontenc}
\usepackage{tikz}

\title{Rapport du projet de programmtion 2}
\author{Jules Kozolinsky, Gabriel Lebouder, Julien Rixte}

\begin{document}
\maketitle

\section{Introduction}

Nous avons subdivisé notre code en différents packages traitant de différents aspects du problème.·\\  Nous allons présenter dans un premier temps les interractions entre nos packages.\\  Nous présenterons ensuite un par un les packages.

\section{Interractions entre nos packages}

\section{Les différents packages}

\subsection{Entities}
Cette partie contient les définition des objets qui vont évoluer sur la map.\\
Ils sont subdivisés en classes comme montré ci-haut.\\
Les classes apparaissant plus haut sont toutes abstraites, on trouve en feuilles des arbres des classes instanciables (Tower1, Tower2, Monster5, Tower1Type, Monster6Type...).\\
les classes XXX1Type sont des objects
\subsection{Game}
\subsubsection{Round}
\subsubsection{Level}

\subsection{Parser}

\subsection{Map}
bonjour
messieurs jules et alix ça va bien ??

tututu
\subsection{Gui}
\subsubsection{Gui}
\subsubsection{Grid}

\section{Ce qu'on aimerait faire}

\end{document}